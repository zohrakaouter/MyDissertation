\documentclass[]{jot} 
% Use the documentclass option 'lineno' to view line numbers

% Enter the JOT metadata in the following 

\usepackage{multirow}
\newcommand{\LLM}{ChatGPT }
\usepackage{tikz}
\usepackage[ruled,vlined,linesnumbered,algo2e]{algorithm2e} 
\usepackage[compatible]{algpseudocode}
\usepackage{tcolorbox} 
\usepackage{rotating}
\makeatletter
% Based on https://tex.stackexchange.com/questions/7032/good-way-to-make-textcircled-numbers
%
\usepackage{pict2e,picture}
\pgfkeys{/csteps/inner ysep/.initial=4pt,
    /csteps/inner xsep/.initial=4pt,
    /csteps/inner color/.initial=black,
    /csteps/outer color/.initial=black,
}
\newsavebox\csteps@CBox
\newlength\csteps@XLength \newlength\csteps@YLength \newlength\csteps@YDepth \newlength\csteps@tmplen
\def\csteps@CircledParam#1#2{\sbox\csteps@CBox{#2}%
    \csteps@XLength=\wd\csteps@CBox\advance\csteps@XLength by\pgfkeysvalueof{/csteps/inner xsep}\relax
    \csteps@tmplen=\pgfkeysvalueof{/csteps/inner ysep}\relax
    \csteps@YDepth=\dp\csteps@CBox\advance\csteps@YDepth by 0.5\csteps@tmplen\relax
    \csteps@YLength=\ht\csteps@CBox\advance\csteps@YLength by\dp\csteps@CBox\advance\csteps@YLength by\pgfkeysvalueof{/csteps/inner ysep}\relax
    \typeout{DBG:#2\space X\space\the\csteps@XLength\space Y:\the\csteps@YLength\space D:\the\csteps@YDepth}%
    \raisebox{-#1\csteps@YDepth}{%
    \ifdim\csteps@XLength>\csteps@YLength
    \makebox[\csteps@XLength]{% X bigger than Y
        \makebox(0,\csteps@YLength){%
            \color{\pgfkeysvalueof{/csteps/outer color}}\put(0,0){\oval(\csteps@XLength,\csteps@YLength)}%
        }%
    \makebox(0,\csteps@YLength){%
        \put(-.5\wd\csteps@CBox,0){\textcolor{\pgfkeysvalueof{/csteps/inner color}}{#2}}%
    }}%
    \else
    \makebox[\csteps@YLength]{%
        \makebox(0,\csteps@YLength){%
            \color{\pgfkeysvalueof{/csteps/outer color}}\put(0,0){\circle{\csteps@YLength}}%
        }%
    \makebox(0,\csteps@YLength){%
        \put(-.5\wd\csteps@CBox,0){\textcolor{\pgfkeysvalueof{/csteps/inner color}}{#2}}%
     }}%
    \fi
    }%
}
\def\Circled#1{\csteps@CircledParam{1}{#1}}
%\def\CircledTop#1{\csteps@CircledParam{0}{#1}}


%\newcommand{\red}[1]{{\color{red}{#1}}}
\newcommand{\red}[1]{{#1}}

\jotdetails{
    volume=23,      % volume
    number=3,       % number or issue
    articleno=a6,   % article number, eg a1 for research articles, e for editorials
    year=2024,      % year
    license=ccbyncnd    % choose from ccby, ccbynd, ccbyncnd
}



\newcommand{\command}[1]{{\color{codepurple}\texttt{\textbackslash #1}}}
\newcommand{\param}[1]{{\color{blue}\texttt{#1}}}
\newcommand{\ie}{\emph{i.e.,}\xspace}
\newcommand{\eg}{\emph{e.g.,}\xspace}
\newcommand{\cf}{\emph{cf.}\xspace}
\newcommand{\etal}{\emph{et al.}\xspace}
\newcommand{\etc}{\emph{etc.}\xspace}
\usepackage{underscore}
% Select the article type

\articletype{regular} 

    % {editorial} editorial 
    % {regular} regular contribution
    % {manual} manual
    % {column} column
   % \usepackage{listings}
\usepackage{graphicx,color,soul}
\usepackage{soul}
\setul{0.5ex}{0.3ex}
\definecolor{Red}{rgb}{1,0.0,0.0}
\setulcolor{Red}

\definecolor{forestgreen}{rgb}{0.0, 0.27, 0.13}
\definecolor{plum}{rgb}{0.56, 0.27, 0.52}
\lstdefinelanguage{Java}{
	morekeywords={abstract, assert, boolean, break,
		byte, case, catch, char,
		class, const, continue, default,
		do, double, else, enum,
		extends, final, finally, float,
		for, goto, if, implements,
		import, instanceof, int, interface,
		long, native, new, package,
		private, protected, public, return,
		short, static, strictfp, super,
		switch, synchronized, this, throw,
		throws, transient, try, void,
		volatile, while, true, false, null},
	keywordstyle=[2]{\textbf},
	keywordstyle={\textbf},
	keywordstyle=\color{plum}\bfseries,
	morecomment=[l]{//}, 
	morecomment=[s]{/*}{*/}, 
	commentstyle=\color{forestgreen},
	morestring=[b]",
	tabsize=4,
	basicstyle=\ttfamily\footnotesize,
	numbers=left,
	numbersep=1pt,
	numberstyle=\footnotesize\color{darkgray},
	stepnumber=1}%TODO basicstyle=\footnotesize
	
	\makeatletter
\def\lst@makecaption{%
  \def\@captype{table}%
  \@makecaption
}
\makeatother
\algnewcommand\algorithmicswitch{\textbf{switch}}
\algnewcommand\algorithmiccase{\textbf{case}}\algdef{SE}[SWITCH]{Switch}{EndSwitch}[1]{\algorithmicswitch\ #1\ \algorithmicdo}{\algorithmicend\ \algorithmicswitch}%
\algdef{SE}[CASE]{Case}{EndCase}[1]{\algorithmiccase\ #1}{\algorithmicend\ \algorithmiccase}%
\algtext*{EndSwitch}%
\algtext*{EndCase}%

\newcommand{\lstETL}[1]{\lstinline[language=ETL,breaklines=true,basicstyle=\listingsfontinline,mathescape,literate={\-}{}{0\discretionary{-}{}{}}]§#1§}

\lstset{
	escapeinside={(*}{*)}
}

\title{An Empirical Study on Leveraging LLMs for Metamodels and Code Co-evolution}

\author[$\ast$]{Zohra Kaouter~Kebaili}
\author[$\ast$]{Djamel Eddine~Khelladi}
\author[$\dagger$]{Mathieu~Acher}
\author[$\ddagger$]{Olivier~Barais}
%\author[$\ast\ast$,2,3]{Sixth Author}

\affil[$\ast$]{CNRS, Univ. Rennes, IRISA, INRIA, France}
\affil[$\dagger$]{INSA Rennes, IUF, IRISA, INRIA, France}
\affil[$\ddagger$]{Univ. Rennes, IRISA, INRIA, France}
%\affil[$\S$]{Author four affiliation}
%\affil[$\ast\ast$]{Author five affiliation}

%\keywords{Typesetting, \LaTeX\, JOT format style.}
\keywords{Metemodel evolution, code evolution, LLM, chatgpt, coevolution, prompt engineering}

\runningtitle{An Empirical Study on Leveraging LLMs for Metamodels
and Code Co-evolution} % For use in the internal pages 

%% For the footnote.
%% Give the last name of the first author if only one author;
% \runningauthor{FirstAuthorLastname}
%% last names of both authors if there are two authors;
% \runningauthor{FirstAuthorLastname and SecondAuthorLastname}
%% last name of the first author followed by et al, if more than two authors.
\runningauthor{Kebaili \textit{et al.}}

\begin{abstract}
\justify
Metamodels play an important role in MDE and in specifying a software language.
%A metamodel is 
They are cornerstone to generate other artifacts of lower abstraction level, such as code. Developers then enrich the generated code to build their language services and tooling, e.g., editors, and checkers. 
When a metamodel evolves, part of the code is regenerated and all the additional developers' code can be impacted. Thus, requiring erroneous code to be co-evolved accordingly. 
%While state of the art at best support a semi-automatic code co-evolution when metamodels evolves. 

In this paper, we explore a novel approach to mitigate the challenge of metamodel evolution impacts on the code using LLMs. In fact LLMs stand as promising tools for tackling increasingly complex problems and support developers in various tasks of writing, correcting and documenting source code, models, and other artifacts. However, while there is an extensive empirical assessment of the LLMs capabilities in generating models, code and tests, there is a lack of work on their ability to support their maintenance. In this paper, we focus on the particular problem of metamodels and code co-evolution. 
%
We first designed a prompt template structure that contains contextual information about metamodel changes, the abstraction gap between the metamodel and the code, and the erroneous code to co-evolve. To investigate the usefulness of this template, we generated three more variations of the prompts. The generated prompts are then given to the LLM to co-evolve the impacted code. %get and treat its answers.
%we conducted a comparison with the use of quick fixes that represent a usual tool of correcting code errors in an IDE.
%\red{We evaluated our generated prompts with \LLM version x.y on nine Eclipse projects from OCL, Modisco, and Papyrus over several evolved versions of three metamodels. }

We evaluated our generated prompts and other three of their variations with \LLM version 3.5 on seven Eclipse projects from OCL and Modisco evolved metamodels.  
Results show that \LLM can co-evolve correctly 88.7 \% of the errors due to metamodel evolution, varying from 75\% to 100\% of correctness rate.
When varying the prompts, we observed increased correctness in two variants and decreased correctness in another variant. We also observed that varying the temperature hyperparameter yields better results with lower temperatures. 
Our results are observed on a total of 5320 generated prompts. 
Finally, when compared to the quick fixes of the IDE, the generated prompts co-evolutions completely outperform the quick fixes.  
\end{abstract}

\acknowledgment{
The research leading to these results has received funding %from the \emph{RENNES METROPOLE} under grant \emph{AIS no. 19C0330} and 
from the \emph{ANR} agency under grant \emph{ANR JCJC MC-EVO$^{2}$ 204687}.
}




\begin{document}
\maketitle
\urlstyle{rm}

\section{Introduction}
\label{intro}

%\todo{rework intro later on}

Large language models (LLMs) have emerged in the field of natural language processing, exhibiting high aptitude to transform and generate textual data. 
Taking advantage of LLMs is highly dependent on good prompts. The use of LLMs is based on the human capacity of crafting high quality prompts: precise and concise. AI community gave the term "Prompt engineering" to the process of designing and refining prompts~\cite{clariso2023model}.
Since their appearance, LLMs have been applied in different domains of scientific research, such as Software Engineering and Model-Driven Engineering (MDE) \cite{10109345,AbukhalafHK23,liu2023improving,hou2023large,pearce2022asleep,sobania2022choose,ziegler2022productivity,vaithilingam2022expectation,nguyen2022empirical,doderlein2022piloting,
nathalia2023artificial,yeticstiren2023evaluating,guo2023exploring,fu2023chatgpt,kabir2023empirical,chaaben2023towards,camara2023assessment,AbukhalafHK23}.

%Prompt engineering
In MDE, metamodels are cornerstone. They define domain concepts and the relations between them~\cite{cabot2012object}.
The metamodel is used to generate other artifacts, such as models, transformations, constraints, and code. The generated code can be used later as a code API to build editors, debuggers, and other language services and tooling. The evolution of the metamodel represents one of the %major 
challenges encountered in MDE. %the model driven engineering field. 
When the metamodel evolves, the code API is re-generated, and as a consequence, the additional code of the tools built on this code API are impacted and may be broken.
%In literature, several studies delve into examining how the evolution of the metamodel influences the generated artifacts. \red{In particular \cite{kessentini2018integrating,kessentini2019automated,cicchetti2008automating,herrmannsdoerfer2009cope,garces2009managing,wachsmuth2007metamodel} have focused on the co-evolution of metamodel and models, \cite{batot2017heuristic,khelladi2017semi,correa2007refactoring,kusel2015systematic}studied the metamodel and constraints co-evolution, and other on the metamodel and transformations co-evolution \cite{kessentini2018automated,khelladi2018change,garces2014adapting,garcia2013model,kusel2015consistent}. }
However, few approaches have addressed the challenge of metamodels and code co-evolution. 
In particular,~\cite{riedl2014towards,kanakis2019empirical,pham2017bidirectional,jongeling2020towards,jongeling2022Structural,zaheri2021towards} focused on consistency checking between models and code, but not its co-evolution. % by repairing the code inconsistencies. 
Other works~\cite{yu2012maintaining,Khelladi2020,khelladi2020power} proposed to co-evolve the code semi-automatically. %However, the former handles only the generated code API, it does not handle additional code and aims to maintain bidirectional traceability between the model and the code API. The latter supports a semi-automatic co-evolution requiring developers' intervention. Moreover, it uses static analysis to propagate the metamodel changes through the additional code. Besides that, that it uses code static analysis and code transformation as an attempt to tackle the metamodel and code coevolution problem.
While LLMs have been so far empirically evaluated to generate qualitative code, refining it, repairing it if vulnerable or augment it \cite{10109345,AbukhalafHK23,liu2023improving,hou2023large,pearce2022asleep,sobania2022choose,ziegler2022productivity,vaithilingam2022expectation,nguyen2022empirical,doderlein2022piloting,
nathalia2023artificial,yeticstiren2023evaluating,guo2023exploring,fu2023chatgpt,kabir2023empirical}, only few works evaluated LLMs in the context of MDE activities, such as generation of models and constraints \cite{chaaben2023towards,camara2023assessment,AbukhalafHK23}.
However, to the best of our knowledge, no existing study evaluated the LLMs capabilities %to support the co-evolution tasks, in particular 
to support developers in the problem of metamodels and code co-evolution. 

In this paper, we fill this gap. 
We explore a novel approach to mitigate the challenge of metamodel evolution impacts on the code using LLMs. 
%Chatgpt as LLM 
Our approach is based on prompt engineering, where we design and generate natural language prompts to best co-evolve the impacted code due to the metamodel evolution. We first designed a prompt template structure that contains contextual information about metamodel changes, the abstraction gap between the metamodel and the code, and the erroneous code to co-evolve. To investigate the usefulness of this template structure, we generated three more variations of these prompts. The generated prompts are then given to the LLM to co-evolve the impacted code errors.

We evaluated our generated prompts and other three of their variations with \LLM version 3.5 on seven Eclipse \red{Modeling Framework (EMF)} projects from OCL, Modisco, and papyrus with three evolved metamodels. 
Results show that \LLM can co-evolve correctly 88.7\% of the errors due to metamodel evolution, varying from 75\% to 100\% of correctness rate. 
%Results also show that varying the temperature does impact the correctness of \LLM co-evolutions. 
When varying the prompts, we observed increased correctness in two variants and decreased correctness in another variant. We also observed that varying the temperature hyperparameter yields better results with lower temperatures. 
Our results are observed on a total of 5320 generated prompts. 
Finally, when compared to the quick fixes of the IDE, the generated prompts co-evolutions completely outperform the quick fixes. 


The paper is structured as follows.  Section 2 gives a motivating example to illustrate the problem of metamodels and code co-evolution. Section 3 presents our approach for generating prompts.  Section 4 details our followed methodology in this empirical study. Section 5 reports on the obtained results and discusses threats to validity. Section 6 discusses related work, and Section 7 concludes the paper. 

\section{Motivating example}
\label{example}


%To have an overview of the challenge we work on, 
To illustrate the adressed challenge, let us have an example. 
Figure \ref{fig: BMM} shows an excerpt of the  version ~0.9.0 of "Modisco Discovery Benchmark" metamodel. %\footnote{\url{https://git.eclipse.org/r/plugins/gitiles/modisco/org.eclipse.modisco/+/refs/tags/0.12.1/org.eclipse.modisco.infra.discovery.benchmark/model/benchmark.ecore}}.
Modisco is an academic initiative project implemented in the Eclipse platform that has evolved numerous times in the past to support the development of model-driven tools, reverse engineering, verification, and transformation of existing software systems \cite{bruneliere2010modisco,bruneliere2014modisco}.
%consisting of 10 classes in the version of~0.9.0.
Figure \ref{fig: BMM} illustrates  some of the domain concepts \textbf{Discovery}, \textbf{Project}, and \textbf{ProjectDiscovery}  used for the discovery and reverse engineering of an existing software system. 
From these metaclasses, a first code API is generated, containing Java interfaces and their implementation classes, a factory, a package, etc. %(details of created files in table \ref{table: locationofMMelemen}). 
Listing \ref{lis:Modisco_Code_API_V1} shows a snippet of the generated Java interfaces and classes from the metamodel in Figure \ref{fig: BMM}. 

The generated code API is further enriched by the developers with additional code functionalities in the "Modisco Discovery Benchmark" project and its dependent projects as well.
For instance, by implementing the methods defined in metaclasses and advanced functionalities in new classes.
% (\eg language services, tooling, \dots).
Listing \ref{lis:Modisco_Code_External_V1} shows the two classes \texttt{Report} and \texttt{JavaBenchmarkDiscoverer} of the additional code ({\small\boxed{Line~4,8}} in the same project "Modisco Discovery Benchmark" and in another dependent project, namely the "Modisco Java Discoverer Benchmark" project).
%, namely the classes \texttt{CDOProjectDiscoveryImpl} and \texttt{JavaBenchmarkPackageImpl}. 
%
%One of the dependent projects on the Modisco discovery benchmark metamodel is Java Benchmark Discoverer. From this project, as an example, we select 3 classes shown in Listing \ref{lis:Modisco_Code_External_V1}. 
%
%In particular, the classes \texttt{CDOProjectDiscoveryImpl} and \texttt{JavaBenchmarkPackageImpl}, and \texttt{JavaDiscoveredProjectImpl}, which handles information and statistics about CDO and Java project. 
%
%the class \textit{CDOProjectDiscoveryImpl} gives information and statistics about CDO project Discovery. 
%Another dependent class \textit{JavaBenchmarkPackageImpl} that creates and initializes the package' methods %for Package models
%( or Benchmark models ? non). The last class is
%\textit{JavaDiscoveredProjectImpl} that contains different information and statistics about java discovered project. 
%The method \textit{initializePackageContents} (line 12) retrieves a \textit{ProjectDiscovery} instance to initialize the classes of the Java Benchmark Package. The method \textit{eBaseStructuralFeatureID} (line 24) returns the structural identifier of the feature relying on the type of its class.  
%
%From version 0.9.0 to 0.11.0, 
In version~0.11.0, the "Modisco Discovery Benchmark" metamodel evolved with several significant changes, among which the following impacting changes: \emph{1)} Renaming the property \texttt{DicoveryDate} of the class  \texttt{JavaBenchmarkDiscoverer} to \texttt{DiscoveryDate}, and \emph{2)} Moving the property \emph{discoveryTimeInSeconds} from metaclass \texttt{Discovery} to \texttt{DiscoveryIteration}.

%To analyze the consequences of the metamodel evolution on the additional code, we track the impacts of the following metamodel changes:
%\begin{enumerate}%[noitemsep,nolistsep]



    %\item Pulled the reference \textit{discoveries} from metaclass \texttt{DiscoveredProject} to \texttt{Benchmark}.
\begin{comment}
    

\begin{itemize}
    \item \emph{1)} Renaming the property \texttt{DicoveryDate} of the class  \texttt{JavaBenchmarkDiscoverer} to \texttt{DiscoveryDate}. % and \texttt{DiscoveredProject}.
    
   \item \emph{2)} Moving the property \emph{discoveryTimeInSeconds} from metaclass \texttt{Discovery} to \texttt{DiscoveryIteration}. 
\end{itemize} %\DK{maybe choose another rename and move ? averageSaveTimeInSeconds ?}
\end{comment}

\begin{figure}

\centering
\includegraphics[width=0.45\textwidth]{pics/example.PNG}
\caption{Excerpt of Modisco Benchmark metamodel in version 0.9.0.}
\label{fig: BMM}
%\vspace{-5mm}
\end{figure}



After applying these modifications, the code of Listing \ref{lis:Modisco_Code_API_V1} is re-generated from the evolved version of the metamodel, which impacts the existing additional code depicted in Listings \ref{lis:Modisco_Code_External_V1}. 

\begin{lstlisting}[language=Java,breaklines=true,mathescape,literate={\-}{}{0\discretionary{-}{}{}},caption=Excerpt of the generated code in org.eclipse.modisco.infra.discovery.benchmark.,label={lis:Modisco_Code_API_V1}]
 //Discovery Interface
   public interface Discovery extends EObject {
  double getTotalExecutionTimeInSeconds();
  void setTotalExecutionTimeInSeconds(double value);
        ...
        }
 //Project Interface
   public interface ProjectDiscovery extends Discovery {...}
 //DiscoveryImpl Class
   public class DiscoveryImpl extends EObjectImpl implements Discovery {
     public double getTotalExecutionTimeInSeconds() {...}
     public void setTotalExecutionTimeInSeconds(double totalExecTime) {...}
         ...
    }
\end{lstlisting}
%xleftmargin=0.2cm,xrightmargin=0cm,framexleftmargin=+6pt,frame=single,
\begin{lstlisting}[language=Java,breaklines=true,mathescape,literate={\-}{}{0\discretionary{-}{}{}},caption=Excerpt of the additional code V1.,label={lis:Modisco_Code_External_V1}]
  
   public class Report {
    ...
       discovery.(*\ul{setDiscoveryTimeInSeconds}*)(...);
   }
    
   public class JavaBenchmarkDiscoverer extends AbstractModelDiscoverer<IFile> {
    ...
       discovery.(*\ul{setDicoveryDate}*)(new Date());
    ...
   } 
\end{lstlisting}


%%%%%%%%%%%%%%%%%%%%%%%%%%%%%%%%%%%%%%%%%%%%%%%%%%%%%%%%%%%
%%                   Now the evolved code                %%
%%%%%%%%%%%%%%%%%%%%%%%%%%%%%%%%%%%%%%%%%%%%%%%%%%%%%%%%%%%
\setulcolor{green} 
%\setstcolor{green}
%\setstcolor{green}
%,xleftmargin=0.2cm,,xrightmargin=-0cm,framexleftmargin=+6pt,frame=single
\begin{lstlisting}[language=Java,breaklines=true,mathescape,literate={\-}{}{0\discretionary{-}{}{}},caption=Excerpt of the additional code V2.,label={lis:Modisco_Code_External_V2}]
  
   public class Report {
    ...
       discovery.(*\ul{getIterations().get(0).}*) 
                    (*\ul{setDiscoveryTimeInSeconds}*)(...);
    ...
   }
   
   public class JavaBenchmarkDiscoverer extends AbstractModelDiscoverer<IFile> {
    ...
       discovery.(*\ul{setDiscoveryDate}*)(new Date());
    ...
   }
\end{lstlisting}




The resulting errors in the original code in version 0.9.0 are underlined in red in Listing \ref{lis:Modisco_Code_External_V1}. Listing \ref{lis:Modisco_Code_External_V2} presents the final result of the manual developer's co-evolution in version 0.11.0. The co-evolved code is underlined in green. 
% Expected co-evolution ?
The changes \textit{rename} of the property \textit{ DicoveryDate} and the \textit{move} of the property \emph{discoveryTimeInSeconds} impact their usages ({\small\boxed{Line~4,8}} in Listing~\ref{lis:Modisco_Code_External_V1}). The impact of renaming \textit{ DicoveryDate} is co-evolved by replacing \textit{setDicoveryDate} by \textit{setDiscoveryDate}. The impact of moving the property \emph{discoveryTimeInSeconds} is co-evolved by extending the call path of the method \emph{setDiscoveryTimeInSeconds} through the reference \textit{iterations} by calling the method \textit{getIterations} and getting the first element of the returned list of DiscoveryIteration objects.

%The above examples show the importance of correctly matching the different code usages of the generated code with the metamodel evolution changes to co-evolve them with the appropriate resolutions.  
Developers unfortunately manually co-evolve the code, which is tedious, error-prone, and time-consuming. 
One help developers get is from the IDE and the provided quick fixes. For example, when using Eclipse quick fixes to co-evolve these errors, it suggests creating the method \texttt{setDiscoveryTimeInSeconds} in the class \texttt{Discovery}, which does not meet the required co-evolutions shown in Listing~\ref{lis:Modisco_Code_External_V2}.

With the ever-growing popularity and promising results of LLMs, a developer can prompt an LLM to suggest a co-evolution. 
For example, 
%To motivate more our work, 
we asked ChatGPT to co-evolve the error resulted from moving the property \emph{discoveryTimeInSeconds} by giving the erroneous code ({\small\boxed{Line~4}} in Listing~\ref{lis:Modisco_Code_External_V1})  with the message of the error taken from eclipse Problems window. This is the first intuition when using ChatGPT because the developer does not know necessarily the metamodel change causing the error and finding it due to the abstraction gap is a tedious and error-prone task. Figure \ref{fig: chatgptanswer} shows that ChatGPT proposes to create a method named \emph{setDiscoveryTimeInSeconds} in the class \emph{Discovery}, which is totally wrong because it does not fit the causing change. 

Our Hypothesis is that the LLM fails because our problem is more complex than simply repairing a code error. It must understand the original impacting metamodel change traced to the code error, as well as the abstraction gap between the two artefacts of metamodels and code. After improving the prompt, \LLM succeeded to give the right resolution as show in Figure \ref{fig: chatgptimprovedanswer}.
Our vision is that this contextual rich information must be injected in the prompt.
Thus, the quality of the prompt is a key for the LLM to solve this problem of metamodels and code co-evolution. %Indeed, it is a hard problem due to the abstraction gap between the two artefacts of metamodels and code. This context should be part of the prompt as well as the impacting metamodle change that must be traced to the code error. 

%by giving only the errouneous ({\small\boxed{Line~4}} in Listing~\ref{lis:Modisco_Code_External_V1}). Chatgpt answer in this case was totaly wrong. Thus, we injected the change information in the request and asked Chatgpt to coevolve the same error. Figure \ref{fig: chatgptanswer} shows that even adding the change information did not help chatgpt to find the right resolution.
%
The next section presents our contribution for a contextualized information rich prompts-based co-evolution of metamodel and code using LLMs.  


\begin{figure}[t]
\centering
\includegraphics[width=0.48\textwidth]{pics/chatgptprimitiveanswer.png}
\caption{ChatGPT primitive answer to the naive prompt.}
\label{fig: chatgptanswer}
%\vspace{-5mm}
\end{figure}

\begin{figure}[t]
\centering
\includegraphics[width=0.48\textwidth]{pics/chatgptimprivedanswer.png}
\caption{\LLM improved answer with the enriched prompt with contextual information.}
\label{fig: chatgptimprovedanswer}
\vspace{-5mm}
\end{figure}

\section{Approach}
\label{sec_approach}
This section presents our proposed overall approach. It first gives an overview. Then, it describes how to detect the metamodel changes and how to trace their impacts until the tests and map them. Finally, it details our prototype implementation. 




\subsection{Overview}

\red{The overall objective of our approach is to help developers in checking the behavioral correctness of the code co-evolution when metamodels evolve, as the co-evolution may be done incorrectly or in an incomplete way (i.e., referred to as partial co-evolution in \cite{le2021untangling,zaidman2011studying}). 
Several ways exist, such as using formal methods, manual code review, or unit tests, etc. Our scope lies in tracing the impact of the metamodel changes till the tests and rely on them as an indicator for behavioral correctness of the code co-evolution, similarly as in a regression testing method \cite{leung1989insights,yoo2012regression,wong1997study}. Our vision is rather than letting the developers execute all test suite in both versions and manually analyzing them, we can reduce the set of tests to be analyzed to the only minimum necessary one. Thus, saving effort and time for developers.}  

%As depicted in 
Figure \ref{fig:appraoch} depicts the overall approach workflow. We first compute the difference between the two metamodel versions each of them having a generated code and an additional code (step {\small\boxed{1}}). In the original version, the additional code is the impacted one, and in the evolved version, the additional code is the co-evolved one. 
After that, we run the impact and the test tracing analysis to link the metamodel changes to the impacted and co-evolved code and their respective tests (step {\small\boxed{2}}). Therefore, a developer can run the traced tests before and after the code co-evolution to check their behavioral correctness. Finally, to ease this task, we map the traced tests and execute them to report them back in a form of a diagnostic to the developers for an easier in-depth analysis of the effect of metamodel evolution rather than analyzing the whole test suite (step {\small\boxed{3}}). 
\blue{Therefore, in a nutshell, there are no particular preconditions to our approach except having available code and tests from both before and after co-evolution along with the delta of the metamodel changes.} 


\begin{figure*}[tb]
\centering
% \hspace*{-2em}
\includegraphics[width=1\textwidth]{img/OverallApproachV2.png}
\caption{Overall approach}
\label{fig:appraoch}
%\vspace{-5mm}
\end{figure*}


\subsection{Detection of metamodel Changes}\label{sec:changes}


Software artifacts continuously evolve over time~\cite{mens2008introduction}.
As any artifact, metamodels evolve as well.  
Two types of changes are known and considered in the literature for metamodel evolution: \emph{atomic} and \emph{complex} changes \cite{Herrmannsdoerfer2011}. 
Atomic changes are additions, removals, and updates of a metamodel element. Complex changes consist of a sequence of atomic changes combined together~\cite{vermolen_reconstructing_2012,khelladi2015detecting}. For example, move property is a complex change where a property is moved from a source class to a target class. This is composed of two atomic changes: delete a property and add a property \cite{Herrmannsdoerfer2011}. 
Several existing approaches allow to automatically detect metamodel changes between two versions, such as \cite{Alter2015, williams2012searching,cicchetti_managing_2009,langer_posteriori_2013,vermolen_reconstructing_2012,Khelladi2016}.

In this work, we use an interface specification of changes {\small\boxed{1}} that is a connection layer to our test tracing approach with the existing change detection approaches. It basically defines each metamodel change as a class with its necessary information. Therefore, in practice, any detection approach \cite{Alter2015, williams2012searching,cicchetti_managing_2009,langer_posteriori_2013,vermolen_reconstructing_2012,Khelladi2016} can be integrated by bridging its changes to our interface and the rest of our approach can be performed independently. 

\begin{table*}[t]
\caption{List of metamodel changes and how they are traced up to the tests in the original and evolved versions. }
\label{tab:changes}
\centering
\resizebox{14cm}{!} {
 \hspace*{-3em}
\begin{tabular}{lcc}
\toprule
                                    & \multicolumn{2}{c}{Tests treatment} \\ \cmidrule{2-3}
\multicolumn{1}{c}{\multirow{-2}{*}{Metamodel changes}} & In original version (V1)            & In evolved version (V2)            \\ \midrule
$\diamond$ Delete property \emph{p} in class \texttt{C}       &   Search for usages of \emph{p} in \texttt{C}                &        \emph{n/a}          \\ \midrule
$\diamond$ Delete class \texttt{C}          &   Search for usages of \texttt{C}               &      \emph{n/a}            \\ \midrule

$\diamond$ Add property \emph{p} in class \texttt{C}     &        \emph{n/a}  & Search for  usages of\emph{p} in \texttt{C}         \\ \midrule
$\diamond$ Add class \texttt{C}     &      \emph{n/a}       &   Search for usages of \texttt{C}                         \\ \midrule


$\diamond$ Rename element \emph{e} to \emph{e'} in class \texttt{C}	           &  Search for usages of \emph{e} in \texttt{C}    & Search for usages of \emph{e'} in \texttt{C}                \\ \midrule

\begin{tabular}[c]{@{}l@{}}$\diamond$ Change multiplicity of property \\ \emph{p} in class \texttt{C} \end{tabular}           &        \multicolumn{2}{c}{Search for usages of \emph{p} in \texttt{C}}           \\ \midrule

\begin{tabular}[c]{@{}l@{}}$\diamond$ Change type of property \emph{p}  \\from \texttt{S} to \texttt{T}\end{tabular}           &      \multicolumn{2}{c}{Search for usages of \emph{p} in \texttt{C}}     \\ \midrule
           
\begin{tabular}[c]{@{}l@{}}$\diamond$ Move property $p_{i}$ from \\ class \texttt{S} to \texttt{T} through \emph{ref}\\
%$\diamond$ Extract class \texttt{S} to \texttt{T} \\with properties $p_{1},...,p_{n}$ \\ \red{through \emph{ref}}\\
$\diamond$ Extract class of properties $p_{1},$\\$...,p_{n}$ from \texttt{S} to \texttt{T} through \emph{ref}\end{tabular}           &     Search for usages of all $p_{i}$ in \texttt{S}             &      Search for usages of all $p_{i}$ in \texttt{T}             \\ \midrule

\begin{tabular}[c]{@{}l@{}}$\diamond$ Push property $p_{i}$ from \\class \texttt{Sup} to \texttt{Sub$_{1}$},...,\texttt{Sub$_{n}$}\end{tabular}           &    Search for usages of all $p_{i}$ in \texttt{Sup}              &    Search for usages of all $p_{i}$ in all \texttt{Sub$_{i}$}               \\ \midrule

\begin{tabular}[c]{@{}l@{}}$\diamond$ Pull property $p_{i}$ from \\classes \texttt{Sub$_{1}$},...,\texttt{Sub$_{n}$} to \texttt{Sup}  \end{tabular}           &     Search for usages of all $p_{i}$ in all \texttt{Sub$_{i}$}             &    Search for usages of all $p_{i}$ in \texttt{Sup}                 \\ \midrule
    
\begin{tabular}[c]{@{}l@{}}$\diamond$ Inline class \texttt{S} to \texttt{T} \\with properties $p_{1},...,p_{n}$\end{tabular}           &      Search for usages of all $p_{i}$ in \texttt{S}               &      Search for usages of all $p_{i}$ in \texttt{T}          \\ %\midrule

%&                  &                  \\ 
                                    
                                    \bottomrule                 
\end{tabular}
}
\end{table*}


\blue{}In practice, we focus on the impacting metamodel changes that will require co-evolution of the code and not on the non-impacting changes. For example, a delete change or a change of type will impact the code and possibly its behavior that can be observed with its tests. 
However, addition changes, although \red{non-breaking}, can be traced back to their newly added tests. Thus, we also consider them to observe their behavior.
The list of metamodel changes \cite{iovino2012impact,cicchetti_managing_2009} we consider for tracing their impact up to the tests is as shown in the first column of Table \ref{tab:changes}. 

For each version of the tests, we use different information provided by each metamodel change, depending on whether we are tracing the impacted tests in the original or the evolved versions. Columns 2 and 3 of table \ref{tab:changes} detail the treatments of each metamodel change in the original and evolved versions. For example, for a rename element \emph{e} to \emph{e'}, we search for \emph{e} and \emph{e'}, respectively, in the original and evolved versions. Similarly, for the other changes, such as Move, Pull, Push, etc. where the source and target classes are different in the original and evolved versions. Only the impact of delete changes is searched in the original version, while the impact of addition changes is only searched in the evolved version. 





\subsection{Tracing the Impacted Tests}
\label{section: tracing the impacted tests}
\red{}


Our approach traces the impact of metamodel changes up to the test. To do that, we structure the code source to better navigate in it. 
Before starting, we parse the code source including tests and build the Code Call Graph (CCG) at the methods level. It consists of nodes $\mathbb{N}$ that are methods, and edges $\mathbb{E}$ that are calls between methods. 
For a given method, the CCG allows us to retrieve its callers, hence, tracing the call methods recursively up to the tests. %map between the code elements (CE) and their usages in the code (CEU). After that, we link the %, as shown in Table \ref{x}.
After that, we apply Algorithm \ref{algo :impactedTestsDetection} on the built CCG. Overall, for each detected metamodel change, the algorithm computes the list of direct and indirect impacted tests that can be traced to the given metamodel change. 


First, we analyze the AST of the code to identify the code usages of the evolved metamodel element. The information concerning the metamodel element before and after evolution, which is included in the metamodel change (see Table \ref{tab:changes}), allow us to spot the impacted code usages (Line 1). For example, for a rename property \emph{id}, the algorithm will first find its usages, such as \emph{getId()} or \emph{setId()}\footnote{\blue{Note that the knowledge about the generated code elements from the metamodel elements (e.g., getter/setter for EAttribute, class/interface for EClass, etc.) is so far hard-coded in the implementation of Algorithm \ref{algo :impactedTestsDetection}. The mappings must be provided for our approach to be able to trace the tests.}}. 
%
Then, we filter these impacted code usages by keeping only the ones found inside a method declaration. % for example : variable declaration or an If condition clause, etc. 
Let us call the found method declaration using the impacted code \textit{IM()} (Line 5). If \textit{IM()} is a test method, that means that we found a direct impacted test (Line 7). Otherwise, thanks to the CCG, we retrieve \textit{IM()}'s parents \emph{parentsOfIM}, which are all the method declarations invoking IM() (Line 9-17). 
%
Afterwards and recursively, we check each parent of \textit{IM()} if it is a test method or not (Line~14). The process is finished if we reach a method declaration that has no parents in the CCG that is either a test or not, or if the reached method declaration is already treated in the \emph{parentsOfIM}. Therefore, by design, after browsing all the impacted code usages, Algorithm \ref{algo :impactedTestsDetection} traces the list of all impacted tests, without missing any if impacted. \red{It is worth noting that tracing all impacted tests holds syntactically and w.r.t. static semantics. Possible side-effect will require further advanced dynamic analysis and is left for future work.} 
Listing \ref{lis:Modisco_impactedtest_V2} presents an example of an impacted test in the Eclipse Modisco.infra.discovery.benchmark project. 
\red{As described in Section \ref{sec:changes}, we detect that the attribute setTotalExecutionTimeInSeconds is renamed and moved from the class Discovery to the class DiscoveryIteration. After that, Algorithm \ref{algo :impactedTestsDetection} detects the code usage \emph{getDiscoveryTimeInSeconds}. Then, it traces it to the method \emph{test000}. As it has the \emph{@Test} annotation, we conclude that \emph{test000} is an impacted test due to the detected move change. 
%
Note that a test can be impacted by multiple metamodel changes, and one metamodel change can impact many tests. Algorithm \ref{algo :impactedTestsDetection} will detect either cases. 
}

\setulcolor{green} 


\subsection{Mapping of impacted tests}

%this would allow to build a summary table between v1 v2 of the mapped table
After having traced the impacted tests, we further assist developers in analyzing the output of our approach. We provide a diagnostic in a form of a visualized report. This report displays the mapping of impacted tests between the original and evolved versions, along with the verdict of their execution (i.e., pass, fail, and error) and the corresponding impacted change. As an input to generate the diagnostic, the developer selects two impacted traced test classes in the original and evolved versions.
%select two test classes that represent the same impacted test class from the first and second versions. 
The mapping between the two sets of test cases for these classes is performed using a state-of-the-art tool, namely GumTree \cite{falleri2014fine}. It parses both test classes into a tree structure to enable the matching of the test cases. Herein, we distinguish three cases, namely:~1) tests that exist in both versions,~2) tests that exist in the original version but not in the evolved one, and~3) tests that exist in the evolved version but not in the original one.
The impacted tests are then executed programmatically using JUnit runner. To facilitate the analysis and the tracing of impacted tests, we include the corresponding impacting metamodel changes in an additional column of the diagnostic report.  

\definecolor{circlegreen}{HTML}{7ed321}
\begin{algorithm}
\caption{Impacted tests detection}\label{algo :impactedTestsDetection}
\begin{algorithmic}[1]
%\textbf{Input :}{codeCallGraph, change}


\Require codeCallGraph, change
%\Ensure $y = x^n$ 
\State impactedUsages $\leftarrow$ match(AST, change)

\State impactedTests $\leftarrow \phi$ 
\For {(impactedUsage $\in$ impactedUsages )}
    
\State \emph{\textcolor{circlegreen}{ \footnotesize{/* Find the method declaration using impactedUsage*/}}}
            
\State IM $\leftarrow$ getIM(impactedUsage, codeCallGraph) 	       
 \If{(isTest(IM))}
                
             \State   impactedTests.add(IM) 	\emph{\textcolor{circlegreen}{/*If not already added*/}}
    
  \Else
            
             \State parentsOfIM $\leftarrow$ getParents(IM, codeCallGraph)
              
            \State  nextRound.add(parentsOfIM)
                        
            \While{(nextRound.hasNewIMs())}
                    
                    \State    IM $\leftarrow$ nextRound.get()
                        
                        
                            \If{(isTest(IM)}
                            
                             \State  impactedTests.add(IM)\emph{\textcolor{circlegreen}{/*If not already added*/}}
                            
                            \Else 
                            
                            \State  parentsOfIM $\leftarrow$ getParents(IM, codeCallGraph)
                             
                           \State   nextRound.add(parentsOfIM) 
                            \EndIf


                        
                    
            \EndWhile
            \EndIf
\EndFor
\end{algorithmic}
\end{algorithm}


\red{Figure \ref{fig:toolView} illustrates a screenshot of the test tracer report on a toy example "Employee Management Project". After selecting the class of tests that have been traced before code co-evolution, and the class of tests that have been traced after code co-evolution, the user clicks on "Map tests" to display the table of mapped tests with their verdict of execution.
 To illustrate the verdict of the test execution, we made: the passing test in \textcolor{green}{green}, the failing test in \textcolor{blue}{blue}, and erroneous tests in \textcolor{red}{red}. For example, the change "Delete Class Contact" impacts two tests, test11 which passes, and test06 which fails. The verdict of the execution of the tests has no relation with the change itself but with the test that uses the code elements impacted by the metamodel change. Those tests do not exist anymore in the evolved version since the class Contact is absent and cannot be tested anymore.}




\subsection{Tool implementation}

We implemented our solution as an Eclipse Java plugin handling Ecore/EMF metamodels and their Java code. %a state-of-the-art
We rely on our approach \cite{khelladi2015detecting} to perform the detection of the metamodel changes bridged with the change interface. 
The test tracing, technically, consists of parsing the java code and manipulating its AST using JDT eclipse plugin\footnote{Eclipse Java development tools (JDT): \url{https://www.eclipse.org/jdt/core/}} to construct the Code Call Graph (CCG). After that, we navigate within the methods calls until we either reach a test or not. Finally, for each impacted \emph{TestClass}, we create a copy \emph{TestClass\_Impacted} where we only include the impacted traced test cases. The developer can then launch the traced tests in both the original and evolved versions to investigate the behavioral correctness of the co-evolved code. 
%The tests' AST nodes are manipulated with edit actions %: modified, replaced or removed)  
%using the package \textit{ org.eclipse.jdt.core.dom.rewrite} and \textit{org.eclipse.core.filebuffers}. 
%\red{say more or reformulate above}
%
We further implemented the diagnostic report as a view that shows the mapped tests with GumTree~\cite{falleri2014fine}, their verdict retrieved programmatically using JUnit Runner\footnote{\url{https://junit.org/junit4/javadoc/4.13/org/junit/runner/Runner.html}} and impacting metamodel changes, as shown in Figure~\ref{fig:toolView}. The goal is to ease the developers' in-depth analysis of the effect of metamodel evolutions rather than rerunning and analyzing the whole test suite.

\begin{figure*}[tb]
\centering
%\hspace*{-1em}
% \vspace{-5mm}
\includegraphics[width=1\textwidth]{img/TestTraceReport.png}
%\caption{Artifacts and structure of a software language in the Eclipse platform.}
\caption{A snippet of the diagnostic report view to visualize and analyze the traced impacted tests.}
\label{fig:toolView}
%\vspace{-5mm}
\end{figure*}

\section{Methodology}
\label{evaluation}

This section describes our methodology for empirically assessing the capabilities of \LLM in addressing the problem of metamodel and code co-evolution. 
It first describes the selected LLM and the followed evaluation process.
Then, it presents our research questions and the data set.

\subsection{Selected LLM}\label{selectedLLM}
%\red(GPT-3.5, 4.0 ?)
% Motivate the use of chatgpt
% Motivate the use of gpt-3.5-turbo, chat mode

We chose to use \LLM \emph{GPT-3.5-turbo} in chat mode. It currently points to \emph{gpt-3.5-turbo-0613} released in June 2023. We opted for this model because of four factors. The first one is that prompt content is only textual, we don't need to inject audio or image content. The second one is its capacity to generate good answers for requests about code and models generation \cite{nathalia2023artificial,yeticstiren2023evaluating,guo2023exploring,fu2023chatgpt,kabir2023empirical,chaaben2023towards,camara2023assessment}. The third factor is that it was the latest API version that is accessible, \emph{gpt-4} API still not accessible for us. The last one is related to the high popularity of \LLM as a tool. It has more than 100 million users, and its website saw more than 1.7 billion visitors in the last three months, with Software and Software Development as visitors' top category\footnote{\url{https://www.similarweb.com/fr/website/chat.openai.com/\#demographics}}. 


\subsection{Evaluation Process}


First, as we aim to query \LLM to co-evolve the erroneous code due to metamodel evolution, we need to provoke the errors in the code. To do so, we replace the original metamodel by the evolved metamodel. Then, we regenerate the code API with EMF. This will cause errors in the additional code that must co-evolve. 
%
After that, we must map the errors with the causing metamodel change before to generate a prompt with all appropriate information to be able to co-evolve the code errors. We then rely on the OpenAI API to query \LLM before to analyze its results. 
%
%\todo{explain how we measure correctness ?}
%
Finally, we measure the correctness of \LLM co-evolution by comparing its co-evolution with the manually co-evolved version by developers. \red{Note that the comparison is processed manually by authors}. This allows us to measure the \emph{correctness} reached by \LLM. %We distinguish between \emph{syntactically} and \emph{semantically} correct co-evolution. The former is when \LLM gives the exact expected co-evolution that is also , whereas the latter is syntactically different but behave the same as the expected co-evolution, hence, semantically correct. 
%We consider a solution to be correct if it matches 
Correctness varies from~0 to~1, i.e., 0\% to 100\% and is defined as follows:
%alternative to terminology of syntactically, semantically
\vspace{0.5em}

\noindent $ Correctness = \dfrac{LLM Coevolutions \cap Manual Coevolutions}{Manual Coevolutions} $

\vspace{0.5em}

We chose the structure shown in Figure \ref{fig:promptstructure} since it contains the contextual information needed for our problem of metamodels and code co-evolution. \red{This structure was built after few manual naive attempts with \LLM, starting from a minimum context (as shown in the Motivating Example of Figure~\ref{fig: chatgptanswer} that failed) and by enriching the structure with more context information}. However, we do not claim its completeness in terms of needed information or if it is the best structure. Other variants can lead to different results. 
%for generated prompts after several manual attempts with Chatgpt. 
To investigate more this choice, we generated three more variations of this structure to observe its effect on the results. Table \ref{table: op variations} contains each variation and its corresponding explanation. Since our prompt contains three parts, we can change the order of these parts (Order Change operator) and the size of the erroneous code we put in the prompt (Minimal Code operator). Finally, rather than asking for a single co-evolution solution, we ask for alternative ones (Alternative Answers operator) in the prompt prefix. \red{To stress out our evaluation, we conducted 5 runs separated by almost a day, the time we needed for one run to check manually the results of each project, which means that each generated prompt is proposed to \LLM five times. This aims to check whether \LLM gives a same or different answer, hence, assess its robustness. 
Finally, we compare \LLM to a baseline, namely the IDE quick fixes that are provided to repair the code errors. Note that we do not take as a baseline \LLM with prompts that only contain the code error, since as shown in Section \ref{example} and Figure \ref{fig: chatgptanswer}, it does not work. 
}


\subsection{Research Questions}

To assess the capabilities of \LLM in the co-evolution of code, we set the following research questions. 

\begin{itemize}
    \item[RQ1] %\subsubsection{RQ1} 
    %Can our code and metamodel coevolution appraoch using LLMs coevolve the code correctly? 
    To what extent can \LLM co-evolve the code with evolved metamodels? 
    This aims to assess the ability of \LLM to give correct resolutions to co-evolve the code according to the metamodel changes.
    
    \item[RQ2] %\subsubsection{RQ2}
    How does varying the temperature hyperparameter affect the output of the co-evolution? %How does our coevolution approach react to the variation of the prompt? 
    The temperature hyperparameter controls the creativity of the language model. 
    This question aims to assess the capability of \LLM to co-evolve the erroneous code when given less or more creativity in the generation of the solution. 
    
    
    \item[RQ3] %\subsubsection{RQ2}
    How does varying the prompt structure affect the output of the co-evolution? %How does our coevolution approach react to the variation of the prompt? 
    This aims to assess the quality improvement of the co-evolutions due to the prompts' variations.
    

    \item[RQ4] %\subsubsection{RQ3} 
    How does \LLM proposed co-evolution compare to the quick fix baseline? 
    As quick fixes are provided by default in an IDE to repair the code errors, this question aims to assess which method outperforms the other in the task of code co-evolution with evolving metamodels. %We measure its precision and recall and compare it with precision and recall of our automatic co-evolution approach.

    
\end{itemize}



\begin{table*}[t]
\centering
\caption{Variation operators of our original prompt (OP).}
\label{table: op variations}
\resizebox{14cm}{!} {
\begin{tabular}{ll}
\toprule
Variation Operators       & Explanation                               \\ \midrule
Order Change (OC)         & \begin{tabular}[c]{@{}l@{}}We change the order between the three structured parts of the prompt. \\  We start by describing  the metamodel change before the abstraction gap.\end{tabular} \\ \midrule
Minimal Code (MC)         & \begin{tabular}[c]{@{}l@{}}Instead of giving the whole method that contains the code error to co-evolve, \\ we only give the instruction of code error.\end{tabular}     \\ \midrule
Alternative Answers (AA)  & \begin{tabular}[c]{@{}l@{}}Instead of asking the LLM to give one solution of co-evolution, \\ we specifically ask for alternative ways to co-evolve the code error.\end{tabular}           \\
\bottomrule

\end{tabular}
}
\end{table*}


\begin{table*}[t]
\centering
		\caption{Details of the metamodels and their evolutions.}
		\label{CaseStudies_Evolution}
	\resizebox{16.5cm}{!} {
	\begin{tabular}{cllll}
		\toprule
		Case study                                                                            & \begin{tabular}[c]{@{}l@{}}Evolved \\ metamodels\end{tabular}                 & Versions & 
		\multicolumn{1}{c}{\begin{tabular}[c]{@{}c@{}}Atomic changes \\in the metamodel\end{tabular}} & \multicolumn{1}{c}{\begin{tabular}[c]{@{}c@{}}Complex changes \\in the metamodel\end{tabular}} \\ \midrule		

	
		OCL & \begin{tabular}[c]{@{}l@{}}Pivot.ecore in project\\ocl.examples.pivot\end{tabular}                        &  \begin{tabular}[c]{@{}l@{}}3.2.2 to\\ 3.4.4\end{tabular}        &   
		\begin{tabular}[c]{@{}l@{}} Deletes: 2 classes, 16 properties, 6 super types \\ Renames: 1 class, 5 properties \\ Property changes: 4 types; 2 multiplicities \\ Adds: 25 classes, 121 properties, 36 super types  \end{tabular}                                                         &    \begin{tabular}[c]{@{}l@{}} 1 pull property \\ 2 push properties  \end{tabular}                                                        \\ \midrule 
		Modisco & \begin{tabular}[c]{@{}l@{}}Benchmark.ecore in project\\modisco.infra.discovery.benchmark\end{tabular}                & \begin{tabular}[c]{@{}l@{}}0.9.0 to\\ 0.13.0\end{tabular}         &     
		\begin{tabular}[c]{@{}l@{}} Deletes: 6 classes, 19 properties, 5 super types \\ Renames: 5 properties  \\ Adds: 7 classes, 24 properties, 4 super types  \end{tabular}                                                        &     \begin{tabular}[c]{@{}l@{}} 4 moves property \\ 6 pull property \\ 1 extract class \\ 1 extract super class \end{tabular}                                                       \\ \midrule		

	 
		Papyrus & \begin{tabular}[c]{@{}l@{}}ExtendedTypes.ecore in project\\papyrus.infra.extendedtypes\end{tabular}            & \begin{tabular}[c]{@{}l@{}}0.9.0 to\\ 1.1.0\end{tabular}         &   
		\begin{tabular}[c]{@{}l@{}}Deletes: 10 properties, 2 super types \\ Renames: 3 classes, 2 properies \\ Adds: 8 classes, 9 properties, 8 super types  \end{tabular}                                                      &    \begin{tabular}[c]{@{}l@{}} 2 pull property \\ 1 push property \\ 1 extract super class \end{tabular} \\  \bottomrule 		
	\end{tabular}
}
\end{table*}



\begin{table*}[t]
		\caption{Details of the projects and their caused errors by the metamodels' evolution.}
		\label{CaseStudies_CoEvolution}
	\resizebox{18.5cm}{!} {
	\begin{tabular}{llcccccc}
		\toprule
		\begin{tabular}[c]{@{}l@{}}Evolved \\ metamodels\end{tabular}                 & \begin{tabular}[c]{@{}l@{}}Projects to co-evolve in response to the \\ evolved metamodels \end{tabular} 	& \begin{tabular}[c]{@{}c@{}}$N^{o}$ of \\ packages\end{tabular} & \begin{tabular}[c]{@{}c@{}}$N^{o}$ of \\ classes\end{tabular} & 
		\begin{tabular}[c]{@{}c@{}}$N^{o}$ of \\ LOC\end{tabular} & \begin{tabular}[c]{@{}c@{}}$N^{o}$ of Impacted \\ classes\end{tabular} & \begin{tabular}[c]{@{}c@{}}$N^{o}$ of total \\ errors \end{tabular}\\ \midrule		

		\begin{tabular}[c]{@{}l@{}}OCL  Pivot.ecore\end{tabular}                     
  & \begin{tabular}[c]{@{}l@{}} $[P1]$ ocl.examples.xtext.base \\ \end{tabular}   
  &   \begin{tabular}[c]{@{}l@{}}12 \end{tabular}    
  & \begin{tabular}[c]{@{}l@{}}181 \end{tabular}                 
  & 	\begin{tabular}[c]{@{}l@{}}17599 \end{tabular} 
  & \begin{tabular}[c]{@{}l@{}} 10 \end{tabular} 
  & \begin{tabular}[c]{@{}l@{}} 29 \end{tabular}
\\ \midrule	
  \begin{tabular}[c]{@{}l@{}}Modisco \\ Benchmark.ecore\end{tabular}            
 & \begin{tabular}[c]{@{}l@{}} $[P2]$ modisco.infra.discovery.benchmark\\ $[P3]$ gmt.modisco.java.discoverer.benchmark\\ $[P4]$ modisco.java.discoverer.benchmark\\ $[P5]$ modisco.java.discoverer.benchmark.javaBenchmark\end{tabular}  
 & \begin{tabular}[c]{@{}l@{}}3 \\8 \\10 \\3  \end{tabular}
 & \begin{tabular}[c]{@{}l@{}}28 \\21 \\28 \\16 \end{tabular} 
 & \begin{tabular}[c]{@{}l@{}}2333 \\1947 \\2794 \\1654 \end{tabular} 
 & \begin{tabular}[c]{@{}l@{}}1\\4 \\9 \\9 \end{tabular} 
 & \begin{tabular}[c]{@{}l@{}}6 \\30 \\56 \\73 \end{tabular} 
 \\ \midrule		
\begin{tabular}[c]{@{}l@{}}Papyrus \\ ExtendedTypes.ecore\end{tabular}          
& \begin{tabular}[c]{@{}l@{}} $[P6]$ papyrus.infra.extendedtypes\\   $[P7]$ papyrus.uml.tools.extendedtypes\end{tabular}  & \begin{tabular}[c]{@{}l@{}}8 \\7 \end{tabular}
& \begin{tabular}[c]{@{}l@{}}37 \\15  \end{tabular} 
&\begin{tabular}[c]{@{}l@{}}2057  \\725  \end{tabular} 
& \begin{tabular}[c]{@{}l@{}}8 \\7   \end{tabular}
& \begin{tabular}[c]{@{}l@{}} 44\\28 \end{tabular} 
\\ \bottomrule		
	\end{tabular}
}
\end{table*}




\subsection{Data set}
\label{dataset}
This section presents the used data set in our empirical study, to be found in the attached supplementary material\footnote{\url{https://figshare.com/s/bf35039892799c0e6f34}}. 

We chose \red{Eclipse Modeling Framework (EMF) platform as technological space}, which allows us to build modeling tools and applications based on Ecore metamodels \cite{steinberg2008emf}. %This open source tool, is also adopted in industry, for instance, in SAP company\footnote{\url{https://help.sap.com/docs/SAP_POWERDESIGNER/1cc460ad80f446e6a9d19303919ee269/c7f1d7456e1b1014b1f5de4946b14e20.html?version=16.7.05}} thanks to its architecture and the tools it offers. It is actively updated, the last release was in November, 2023\footnote{\url{https://download.eclipse.org/modeling/emf/emf/builds/release/latest/index.html}}.    % \cite{ref?}. %In 2022, Eclipse IDE was downloaded 1 million times per month.
%
First, we aimed at selecting metamodels with meaningful \red{real} evolutions that do not consist in only deleting metamodel elements, but rather including complex evolution changes (see subsection \ref{mmchanges}). 

%By complex changes, we mean ....

This selection criterion resulted in seven Java projects from three case studies of three different language implementations in Eclipse, namely OCL~\cite{MDTOCL}, Papyrus \cite{MDTPapyrus}, and Modisco~\cite{MDTModisco} \red{with their versions that was manually co-evolved by developers, that represent our ground of truth}.
%
OCL is a standard language defined by the Object Management Group (OMG) to specify First-order logic constraints. Papyrus is an industrial project led by CEA\footnote{\url{http://www-list.cea.fr/en/}} to support model-based simulation, formal testing, safety analysis, etc. Modisco is an academic initiative to support development of model-driven tools, reverse engineering, verification, and transformation of existing software systems. 
%Papyrus is an industrial project led by CEA\footnote{\url{http://www-list.cea.fr/en/}} to support model-based simulation, formal testing, safety analysis, etc.  
Thus, the case studies cover standard, industrial, and academic languages that have evolved several times for more than 10 years of continuous development period.
%In particular, Papyrus and OCL are open source projects are actively maintained with frequent releases per year.

Table~\ref{CaseStudies_Evolution} presents the details on the selected case studies, in particular about their metamodels and the occurred changes during evolution. The total of applied metamodel changes was~330 atomic changes, including~19 complex changes in the three metamodels. 
%
Table~\ref{CaseStudies_CoEvolution} presents the details on the size of the seven projects and code of the original versions that we co-evolve in addition to the number of errors after the metamodels evolution. 



\section{Results}
\label{results}
This section now presents our results answering each RQ. 

\subsection{RQ1: Assessing the ability of \LLM to give correct code co-evolutions }

%execute with temp 0.2, single result, with whole method having the error, on many projects, 

%To what extent can \LLM co-evolve the code with metamodels? This aims 
To assess the ability of \LLM to give correct code co-evolutions, % the code according to the metamodel changes. 
%To answer this question, 
we ran over the errors caused by the metamodel changes and generated 266 prompts that we submitted to \LLM. 
Note that we ran our original prompts asking for a single co-evolution of the code error. We further set a fixed temperature hyperparameter to~0.2. A value of 0 restricts the generation of a solution towards a more deterministic one and the more it is higher the more creative the model gets in generating a different solution. Thus, we chose 0.2 to allow only little creativity while restricting the model to not get a different answer for the same prompts in different executions, since we ask for a single solution. 
%
Among the~266 generated prompts, \LLM gave, on average, a correct code co-evolution in~88.7\% of the time, varying from~75\% to 100\%. 
% \red{75\%} to \red{100\%}. 
\red{When taking only on complex changes into consideration, correctness improves on average from~88.7\% to~95.2\%, \ie \LLM performs better for complex changes.}
Results show a promising ability of \LLM to actually co-evolve code with evolving metamodels when given the right contextual information in the prompts, rather than simply the errors and their messages. 



\begin{table}
        
        \caption{Measured correctness rate per temperature.}
        \label{table:correctnesspertemp}
        %\hspace*{-0.5cm}
        \resizebox{0.45\textwidth}{!} {
        \begin{tabular}{lllllllll}
        \toprule
        \centering
            & \multicolumn{6}{l}{\hspace{10.5em}\bfseries Projects}                                \\ %\cmidrule{3-7} 
            \multirow{3}{*}{\begin{turn}{-90}\bfseries Temperature\end{turn}} & \multicolumn{1}{l|}{}  & \multicolumn{1}{l|}{P1}       & \multicolumn{1}{l|}{P2}      &\multicolumn{1}{l|}{P3} & \multicolumn{1}{l|}{P4}   & \multicolumn{1}{l|}{P5}  & \multicolumn{1}{l|}{P6} & P7 \\ \cmidrule{2-9}
            
            & \multicolumn{1}{l|}{0.0} & \multicolumn{1}{l|}{76\%} & \multicolumn{1}{l|}{62.5\%}& \multicolumn{1}{l|}{100\%}& \multicolumn{1}{l|}{80\%}&\multicolumn{1}{l|}{86\%}&\multicolumn{1}{l|}{93.7\%}& 92.8\% \\ \cmidrule{2-9} 
            
            & \multicolumn{1}{l|}{0.2} & \multicolumn{1}{l|}{\cellcolor{green!35}\textbf{84\%}}& \multicolumn{1}{l|}{\cellcolor{green!35}\textbf{75\%}}& \multicolumn{1}{l|}{\cellcolor{green!35}\textbf{100\%}}& \multicolumn{1}{l|}{\cellcolor{green!35}\textbf{82\%}} &\multicolumn{1}{l|}{\cellcolor{green!35}\textbf{88\%}} &\multicolumn{1}{l|}{\cellcolor{green!35}\textbf{95.8\%}} & \cellcolor{green!35}\textbf{96.4\%} \\ \cmidrule{2-9} 
            
             & \multicolumn{1}{l|}{0.5} & \multicolumn{1}{l|}{57\%}& \multicolumn{1}{l|}{50\%}& \multicolumn{1}{l|}{100\%}& \multicolumn{1}{l|}{58\%} &\multicolumn{1}{l|}{68\%} &\multicolumn{1}{l|}{87.5\%} & 89.2\% \\ \cmidrule{2-9} 
             
              & \multicolumn{1}{l|}{0.8} & \multicolumn{1}{l|}{50\%}& \multicolumn{1}{l|}{37\%}& \multicolumn{1}{l|}{0\%}& \multicolumn{1}{l|}{32\%} &\multicolumn{1}{l|}{46\%} &\multicolumn{1}{l|}{62.5\%} & 85.7\% \\ \cmidrule{2-9} 
              
               & \multicolumn{1}{l|}{1.0} & \multicolumn{1}{l|}{30\%}& \multicolumn{1}{l|}{29\%}& \multicolumn{1}{l|}{0\%}& \multicolumn{1}{l|}{26\%} &\multicolumn{1}{l|}{40\%} &\multicolumn{1}{l|}{68.7\%} & 78.5\% \\ \bottomrule%\cmidrule{2-9} %
        \end{tabular}
        }
    \end{table}


\begin{tcolorbox}[boxsep=-2pt]
\textbf{$\boldsymbol{RQ_1}$ insights:}
Results confirm that \LLM is able at 88.7\% to correctly co-evolve code due to metamodel evolution thanks to the information on the abstraction gap and the impacting metamodel change. It also performs better on complex changes.  
\end{tcolorbox}


\subsection{RQ2: Studying the impact of temperature variation on the output co-evolutions}
%execute with temp 0, 0.5, 0.8, 1.0

The temperature hyperparameter controls the creativity of the language model. The temperature hyperparameter in \LLM can be set between~0 and~2. Yet, it is suggested to only set it between~0 and~1 in the documentation\footnote{\url{https://platform.openai.com/docs/api-reference/chat}}. Thus, to assess the effect of the temperature in co-evolving the code, we will vary it on~0,~0.2,~0.5,~0.8,~and~1.0. 
Table \ref{table:correctnesspertemp} shows the obtained results. 

%How does varying the temperature hyperparameter affect the output of the co-evolution? %How does our coevolution approach react to the variation of the prompt? 

%This question aims to assess the capability of \LLM to co-evolve the erroneous code when given less or more creativity in the generation of the solution. 
%we observe that the more we increase the temperature, the less correct are the code co-evolutions that are suggested by
Overall, we observe that as the temperature increases, the correctness of code co-evolutions suggested by \LLM decreases. Notably, the correctness improves when the temperature is increased from~0 to~0.2. However, it subsequently degrades with the further increase in temperature, up to~1. At temperatures~0.8 and~1, we observed the worst decrease in correctness. % does not exceed~50\% and~40\%, respectively. 
The best performance of \LLM is obtained at the temperature~0.2. Note that even with a temperature set to~0, which implies no creativity, \LLM yields results that are nearly as satisfactory as those obtained with a temperature of~0.2.

\red{Moreover, when we repeated the same prompt for each temerature for 5 times, the five answers of \LLM were similar. \LLM gives almost the same proposition of co-evolution, sometimes it uses different terms to comment its answer. For example: \textit{"// Remove the following line since DiscoveredProject is removed"} and \textit{" // Code using DiscoveredProject should be removed"}. However, they are the same co-evolutions. Sometimes \LLM also uses intermediate variables to give the same co-evolution, for example:\textit{ "return aReport.generate()"} and \textit{benchmarkModel=aReport.generate() return benchmarkModel"}. 
We believe that this result of obtaining the same co-evolutions over 5 runs shows the efficiency of the prompt template in Figure~\ref{fig:promptstructure}. 
By including the necessary information (i.e., abstraction gap, causing change information, and code error), it allows \LLM to narrow the scope of possibilities and to propose similar answer for each unique prompt in each run.}

\begin{tcolorbox}[boxsep=-2pt]
\textbf{$\boldsymbol{RQ_2}$ insights:}
Results show that lower temperature of~0 and~0.2 give better co-evolutions from \LLM with~0.2 being the best we observed. 
Interestingly, we obtained the same results over 5 different runs, which suggests the efficiency our prompt structure in narrowing the scope of possibilities of \LLM's answers. 
\end{tcolorbox}


\subsection{RQ3: Studying the impact of prompt structure variation on the output co-evolutions}
%Execute prompts starting by the gap information followed by the change information then 

%by starting by the change followed by the gap

%execute prompt with minimal code (instruction)

%??? execute prompt with less code (surrounding instruction) ???
%??? execute prompt without abstraction gap ???

%execute prompt while asking for many alternative solutions

%How does varying the prompt affect the output of the co-evolution? %How does our coevolution approach react to the variation of the prompt? 
In our approach, we propose a possible structure for the generated prompts (cf. Subsection \ref{promptgeneration}). However, there is no assurance that this represents the best proposition. Therefore, we varied the structure that we proposed in three different ways, as shown in Table \ref{table: op variations}.  Then we ran the three variations of the original prompts in order to assess the fluctuation and effect on the correctness of the proposed co-evolutions. Here we set the temperature to 0.2 based on results of RQ2. % due to the variation of the prompt. 

%we can of course vary them, which in turn can affect the correctness of the proposed co-evolutions. 
%To assess the quality improvement of the proposed resolutions due to the variation of the prompt. We ran the experiments with three additional variations of our original prompts, as shown in Table \ref{table: op variations}. 

%\red{Table \ref{x} shows xyz relevant prompt variations in our context of metamodels and code co-evolution that we will assess.}


\begin{table}
 \caption{Measured correctness rate for different prompt variations. [$\nearrow$ and  $\searrow$ are increase and decrease in correctness.]}
        \label{table:correctnessVariation}
        \hspace*{-0.5cm}
     \resizebox{0.5\textwidth}{!}{ 
       
        \begin{tabular}{lllllllll}
        \toprule
        \centering
            & \multicolumn{8}{l}{\hspace{15.5em}\bfseries Projects}                                \\ %\cmidrule{3-7}
            \multirow{3}{*}{\begin{turn}{-90}\bfseries Variations\end{turn}} & \multicolumn{1}{l|}{}  & \multicolumn{1}{l|}{P1}       & \multicolumn{1}{l|}{P2}      &\multicolumn{1}{l|}{P3} & \multicolumn{1}{l|}{P4}  &\multicolumn{1}{l|}{P5}  &\multicolumn{1}{l|}{P6}  & P7 \\ \cmidrule{2-9}
            
            %& \multicolumn{1}{l|}{\begin{tabular}[c]{@{}l@{}}Original \\ Prompt (OP)\end{tabular}} & \multicolumn{1}{l|}{84\%} & \multicolumn{1}{l|}{75\%}& \multicolumn{1}{l|}{100\%}& \multicolumn{1}{l|}{82\%}& \multicolumn{1}{l|}{88\%}& \multicolumn{1}{l|}{95.8\%}& 96.4\% \\
            %\cmidrule{2-9} 

            & \multicolumn{1}{l|}{\begin{tabular}[c]{@{}l@{}}Original \\ Prompt (OP)\end{tabular}} & \multicolumn{1}{l|}{\textbf{84\%}}& \multicolumn{1}{l|}{\textbf{75\%}}& \multicolumn{1}{l|}{\textbf{100\%}}& \multicolumn{1}{l|}{\textbf{82\%}} &\multicolumn{1}{l|}{\textbf{88\%}} &\multicolumn{1}{l|}{\textbf{95.8\%}} & \textbf{96.4\%} \\ \cmidrule{2-9} 
            
            & \multicolumn{1}{l|}{\begin{tabular}[c]{@{}l@{}}Order \\ change (OC)\end{tabular}} & \multicolumn{1}{l|}{84\%}& \multicolumn{1}{l|}{\cellcolor{red!15}66\%$\searrow$}& \multicolumn{1}{l|}{100\%}& \multicolumn{1}{l|}{\cellcolor{red!15}74\%$\searrow$} &\multicolumn{1}{l|}{\cellcolor{red!15}86\%$\searrow$} &\multicolumn{1}{l|}{95.8\%} & 96.4\% \\
            \cmidrule{2-9} 
            
             & \multicolumn{1}{l|}{\begin{tabular}[c]{@{}l@{}}Minimal \\ Code (MC)\end{tabular}} & \multicolumn{1}{l|}{\cellcolor{green!35}88.4\%$\nearrow$}& \multicolumn{1}{l|}{\cellcolor{green!35}87.5\%$\nearrow$}& \multicolumn{1}{l|}{100\%}& \multicolumn{1}{l|}{\cellcolor{green!25}86\%$\nearrow$} & \multicolumn{1}{l|}{\cellcolor{green!35}96\%$\nearrow$} & \multicolumn{1}{l|}{\cellcolor{green!25}97.9\%$\nearrow$} & 96.4\% \\
             \cmidrule{2-9} 
             
              & \multicolumn{1}{l|}{\begin{tabular}[c]{@{}l@{}}Alternative \\ Answers (AA)\end{tabular}} & \multicolumn{1}{l|}{84\%}& \multicolumn{1}{l|}{\cellcolor{green!25}79\%$\nearrow$}& \multicolumn{1}{l|}{100\%}& \multicolumn{1}{l|}{\cellcolor{green!35}92\%$\nearrow$} & \multicolumn{1}{l|}{\cellcolor{green!25}92\%$\nearrow$} &\multicolumn{1}{l|}{95.8\%} &96.4\%\\ \bottomrule %\cmidrule{2-7} %
              %\cellcolor{green!35}$\searrow$
              %\textcolor{red}{$\searrow$} \textcolor{green}{$\nearrow$}
            
        \end{tabular}
        }
    \end{table}


Table \ref{table:correctnessVariation} shows the obtained results. Overall, we observe slight increase and decrease compared to the original Prompts structure~(\textbf{OP}). 
We observe that changing the order in the prompt by first describing the metamodel change (\textbf{OC}) decreases the correctness of the proposed \LLM co-evolutions. It decreased by~-9\% in P2,~-8\% in P4 and by~-2\% in P5. We observed that in particular when describing the abstraction gap with the generated elements for the metamodel deleted classes, % information in the case of deletions, 
\LLM assumes that the code classes still exist in the code and were not deleted. Thereby, altering sometimes the correctness of its proposed co-evolutions. 

However, the two other prompt variations of giving a minimal code containing the error and asking for alternative solutions delivered better overall results. 
Surprisingly, on the one hand, giving only the code instruction containing the error (\textbf{MC}) gave the best results. It improved by~+4.4\% in P1,~+12.5\% in P2,~+4\% in P4, ~+8\% in P5, and ~+2.1\% in P6. 
Our observation is that this improvement is sometimes due to the inability of \LLM to find the impacted code element and co-evolve it within complex and long methods. 
On the other hand, when asking for alternatives, it improved only by~+4\% in P2,~+10\% in P4, and~+4\% in P5. We observed that when \LLM is unable to find the correct resolution with (\textbf{OP}) prompts, it is unlikely to find the right one when asking for alternatives (\textbf{AA}). Only in few cases it could find the correct co-evolutions.  %We noticed also that 
The generation of prompts and saving the results took on average from about~10 seconds per prompt for the Original Prompts to~84 seconds per prompt in the case of Alternative Answers (AA) variation prompts.
\red{Finally, when focusing only on complex changes, correctness improves on average from 88.7\% to 97.6\% for \textbf{(MC)} and \textbf{(AA)}. This implies that, overall, \LLM also performs better for complex changes when varying the prompts. This can be explained by the fact that complex changes provide much more context information that guide better \LLM to give better responses.} 
%on  the variation operator



\begin{tcolorbox}[boxsep=-2pt]
\textbf{$\boldsymbol{RQ_3}$ insights:}
Results show an improvement with two variants out of the three we explored. However, gains are not significant compared to our original structure of the prompt. 
Variants also perform better on complex changes. 
\end{tcolorbox}

\subsection{RQ4: Comparison with quick fixes as baseline}
%230441
%compare with quick fixes

%How does \LLM proposed co-evolution compare to the quick fix baseline? 
%As quick fixes are provided by default in an IDE to repairs the code errors, this question aims to assess which method outperform in the task of code co-evolution with evolving metamodels.

To compare to a baseline the obtained results of proposed co-evolutions with our generated prompts, we checked what is the best quick fix an IDE proposes for each error. Algorithm~\ref{algo :quickfixes} shows the followed steps to run the quick fixes on our projects automatically. 
It iterates over the java classes and for each error, we  automatically apply the top quick fix suggested by the IDE. 
It stops when all the errors disappear or if the remaining errors have no possible quick fixes purposed by the IDE. 
%It stops in three cases : 1) if all the errors disappear, or 2) if the remaining errors have no possible quick fixes, or 3) the application of a quick fix causes an infinite loop \cite{cuadrado2018quick,khelladi2019detecting}, i.e., a cycle of applying a quick fix that causes a previously fixed error~(~A $\mapsto$ ~B $\mapsto$ ~A $\mapsto$ ~B $\mapsto$ ...). 

In Table~\ref{AppliedQF}, we present the percentage of errors that quick fixes eliminated for each project. %, and the frequencies of each type of applied quick fixes. 
%
While the quick fixes eliminated from~41\% to~100\% errors. We use the term eliminated instead of correct co-evolution because no quick fix was applied as expected to the manual developers' co-evolutions. In other words, the correctness rate of automatic quick fixes are equal to 0 and are not suited for the task of metamodels and code co-evolution.
For example, concerning errors caused by class or property deletion from the metamodel, renaming a class or an attribute, moving, pushing or pulling attributes or methods from a class to another, the quick fixes proposed to create them back in their old containers. This is in contradiction of the applied metamodel changes and the code co-evolution.  

For errors caused by changing a variable's type, the quick fixes always proposed to add a cast with a wrong type. 

Similarly, as naive prompts with only the code errors and their messages, quick fixes do not take in consideration the knowledge of the abstraction gap and the information contained in its causing metamodel changes. For example, the quick fix \texttt{create the missing method \emph{m()}} is applied no matter the metamodel change (i.e., deletion, moving, pulling, or pushing changes) and no matter the metamodel element it was generated from.
Our approach of generated prompts takes into account the context of the impacted code, the abstraction gap, and the causing metamodel change thanks to the prompt template that we designed (cf. Table \ref{fig:promptstructure}).

\begin{tcolorbox}[boxsep=-2pt]
\textbf{$\boldsymbol{RQ_4}$ insights:}
Results show that \LLM with our generated prompts completely outperforms the quick fixes in correctly co-evolving the code. 
\end{tcolorbox}




\begin{algorithm2e}[t]
% \algsetup{linenosize=\tiny}
 \small
\SetAlgoLined
\KwData{EcoreModelingProject}
javaClasses $\leftarrow$ Parse(EcoreModelingProject)

\For {( jc $\in$ javaClasses)}
{
    errorsList $\leftarrow $ getErrors(jc)
    
    \While{(!errorsList.isEmpty() \& hasQuickFix)}
    {
        error  $\leftarrow$ errorsList.next()
          
          \uIf {error.hasQuickickFix() }
      
        {
        useQuickFixes(error) %\COMMENT{Eclipse quick fix}
        }
        
                 
        refreshJavaClass(jc) 
                
        refreshErrorsList(jc, errorsList)
    }

}
 
 \caption{Quick fixes for coevolution}
 \label{algo :quickfixes}
\end{algorithm2e}



\begin{table}[t]
	\centering
	\caption{Number of applied Quick Fixes for each project and per evolved metamodel.}
	\label{AppliedQF}
	\resizebox{0.38\textwidth}{!} {
		\begin{tabular}{lll}%ll}
			\toprule
			\begin{tabular}[c]{@{}l@{}}Evolved \\ metamodels\end{tabular} & \begin{tabular}[c]{@{}l@{}}Co-evolved \\ projects\end{tabular}  & \begin{tabular}[c]{@{}l@{}} \% of eliminated \\ errors  \end{tabular} %& \begin{tabular}[c]{@{}l@{}} $N^{o}$ of applied \\ Quick Fixes \end{tabular}  & Total
   \\ \midrule 
   \begin{tabular}[c]{@{}l@{}}OCL Pivot.ecore\end{tabular}

			 & $[P1]$ & 41\% 
    %&\begin{tabular}[c]{@{}l@{}} $ [ $Create method $] $: 4,$[ $Change to m' $] $:11,\\     $ [ $change type of var or return type $] $: 1,\\     $ [ $Add unimplemented methods $] $: 2,     \end{tabular}  & 18
    \\ \midrule			
    \multirow{4}{*}{\begin{tabular}[c]{@{}l@{}}Modisco \\ Benchmark.ecore\end{tabular}}  &  $[P2]$& 100\%
    %& \begin{tabular}[c]{@{}l@{}}  
    %  $[ $  $] $     $[ $ Create Class X  $] $: 6,  $[ $ Add unimplemented methods  $] $: 1    \end{tabular}  & 7
    \\ \cmidrule{2-3}			

			& $[P3]$ &100\% 
   %& \begin{tabular}[c]{@{}l@{}}    $[ $ Create Class X $] $: 2,$[ $  Create method $] $: 8,\\ $[ $ Change m to m' $] $; 5, $[ $ Remove argument $] $: 1,\\ $[ $ Add cast $] $ : 6   \end{tabular}  & 22
   \\ \cmidrule{2-3}			

			& $[P4]$ &83\%
   %& \begin{tabular}[c]{@{}l@{}}     $[ $  Create Method $] $: 17,  $[ $ Change method m to m' $] $: 16, \\ $[ $ Remove argument $] $: 2,  $[ $ Change type of var or return type $] $: 1,\\  $[ $ Add Cast$] $: 16.    \end{tabular}& 52 
   \\ \cmidrule{2-3}			

			& $[P5]$&67\% 
   %& \begin{tabular}[c]{@{}l@{}} $[ $ Change method m to m' $] $: 4, $[ $ Add unimplemented methods  $] $: 2,\\ $[ $ Create Constant$] $:9, $[ $ change type $] $ : 2,\\$[ $ Add Cast  $] $: 6    \end{tabular}& 23  
   \\ \midrule			

			\multirow{2}{*}{\begin{tabular}[c]{@{}l@{}}Papyrus \\ ExtendedTypes.ecore\end{tabular}} &  $[P6]$&69\%
   %& \begin{tabular}[c]{@{}l@{}}     $[ $ Create Class X $] $: 3,  $[ $ Create method m $] $: 9,\\  $[ $ Change method m to m' $] $: 13,  $[ $Change type of var or return type  $] $: 5, \\$[ $ Add Cast $] $: 4,  $[ $Add unimplemented methods$] $: 2.   \end{tabular} & 36
   \\ \cmidrule{2-3}			

		
			&  $[P7]$ &93\% 
   
   %\begin{tabular}[c]{@{}l@{}}     $[ $ Create Class X $] $: 1,  $[ $ Create method $] $ : 14,\\ $[ $ Create Constant $] $ :3,  $[ $ Add Cast$] $: 2,\\  $[ $ Change method m to m' $] $:3.   \end{tabular} & 23
   \\ %\midrule			
%Total &&&& 429 \\ 
\bottomrule

		
		\end{tabular}  
  
	}
	%\hspace{1em}
\end{table}

\subsection{Threats to Validity}
This section discusses threats to validity w.r.t. Wohlin et al. \cite{wohlin2012experimentation}.

\subsubsection{Internal Validity.} 
%internal => prompt engineering?? only vary temperature, choice of API Mode completion (our choice) vs Chat

A first internal threat is that we only varied the temperature hyperparameter over \LLM API. The documentation suggests not to modify top\_p and temperature at the same time, so we chose to let the default value of top\_p =~1. 
Moreover, to measure the correctness, we analyzed the developers' manual co-evolution.
To mitigate the risk of misidentifying a manual co-evolution, for each impacted code error, we mapped it in the co-evolved class version. If we did not find it, we further searched in other classes in case the original impacted part of the code was moved into another class. Thus, our objective was to reduce the risk of missing any mappings between an error in the original code and its co-evolved version. Moreover, as our co-evolution relies on the quality of detected metamodel changes. We also validated each detected change by checking whether it occurred between the original and evolved metamodels. This alleviates the risk of relying on an incorrect metamodel change that would degrade the generated prompts and lead to wrong co-evolution from \LLM. 
%Note that the order of changes taken as input does not influence our co-evolution, but the order of errors we treat may affect the distribution of the applied resolutions. However, we took the order in which the errors were detected. 
%\red{Finally, to reduce risk of getting random answers from \LLM, we ran 5 times the evaluation and the prompts. It showed that our results are robust and that the capability of \LLM in co-evolving code is not due to randomness.} 

\subsubsection{External Validity.}
%external => cannot generalize to other LLMs such as Lamma, => future work to replicate

We implemented and ran the empirical study for EMF and Ecore metamodels and Java code. Note that \red{the choice of Java is imposed by EMF and no other languages are supported. Thus, we naturally do not generalize to other languages.} 
We also relied only on \LLM and GPT-3.5-turbo released in june~2023. 
Therefore, we cannot generalize our approach to other LLMs and other future versions of \LLM. It is also unclear how our findings would transfer to other benchmarks other than EMF Ecore metamodels and java code. Further experiments are necessary in future to get more insights and- before any generalization claims.  


\subsubsection{Conclusion Validity.}
%conclusion => do we have significant results 

Our empirical study show promising results with \LLM being able to generate correct code co-evolution when given the necessary contextual information. It showed to be useful, with an average of~88.7\% correctness (from~75\% to~100\%). The results also show the benefit over relying on quick fixes. Nonetheless, even though we evaluated it on real evolved projects, we must evaluate it on more case studies to have more insights and statistical evidence.  %further evaluation is needed on more case studies.

\subsection{Discussion and Limitations}
The rationale behind the prompt structure, which an important part in our empirical study, is that our problem concerns the use of the code generated from the metamodels and their changes and not only error repair. Moreover, our results show that \LLM can co-evolve the code correctly by setting lower temperature, especially in case of complex metamodel changes. Furthermore, repeating the experimentation five time has led to the same results shown in Table \ref{table:correctnesspertemp} and Table \ref{table:correctnessVariation}, which confirms the robustness of \LLM in the code co-evolution task and that his capability is not due to randomness.
Finally, we handled a single metamodel with changes that are independent between them. Treating the case of multiple metamodel and the case of interdependent changes would need setting an order of priority between them, which we left for a future work.
%Overall, our findings contribute to a deeper understanding of \LLM ability in the metamodel and code co-evolution context
%say it works well, remind bets results,  => it could replace quick fixes

%then discuss limitation 1) des LLMs for co-evolution  2) of our prompts 


\section{Related Work}
\label{RelatedWork}

This section discusses close related work that focuses on empirically evaluating LLMs on code and MDE artifacts. % for the task of metamodels and code co-evolution.  


In literature, several studies delve into examining how the evolution of the metamodel influences the generated artifacts. In particular
\cite{kessentini2018integrating,kessentini2019automated,cicchetti2008automating,herrmannsdoerfer2009cope,garces2009managing,wachsmuth2007metamodel} have focused on the co-evolution of metamodel and models, \cite{batot2017heuristic,khelladi2017semi,correa2007refactoring,kusel2015systematic} studied the metamodel and constraints co-evolution, and other on the metamodel and transformations co-evolution \cite{kessentini2018automated,khelladi2018change,garces2014adapting,garcia2013model,kusel2015consistent}. Other approaches focused on the model consistency repair (e.g., \cite{kretschmer2017abstract,kretschmer2021consistent,kretschmer2021transforming,macedo2013model,pinna2015resolving}).  
However, only a few works addressed the challenge of metamodels and code co-evolution. 
In particular, \cite{riedl2014towards,kanakis2019empirical,pham2017bidirectional,jongeling2020towards,jongeling2022Structural,zaheri2021towards} focused on consistency checking between models and code, but not on its co-evolution. % by repairing the code inconsistencies. 
Other works~\cite{yu2012maintaining,Khelladi2020} proposed to co-evolve the code. However, the former handles only the generated code API, it does not handle additional code and aims to maintain bidirectional traceability between the model and the code API. The latter supports a semi-automatic co-evolution requiring developers' intervention. %Moreover, it uses static analysis to propagate the metamodel changes through the additional code. Besides that, that it uses code static analysis and code transformation as an attempt to tackle the metamodel and code coevolution problem.


Furthermore, several works evaluated the use of LLMs in software engineering tasks. 
%
Early studies on Copilot focus on the exploration of the security of the generated code~\cite{pearce2022asleep}, comparison of the performances of Copilot with mutation-based code generation techniques~\cite{sobania2022choose}, and  the impact on productivity and the usefulness of Copilot for developers~\cite{ziegler2022productivity,vaithilingam2022expectation}. 
%
%Nguyen et al. \cite{nguyen2022empirical} considered 33 problems in four languages to generate code for with Copilot.
Nguyen et al.~\cite{nguyen2022empirical} performed an early empirical study on the performance and understandability of Copilot generated code on~34 problems from Leetcode. 
%
Doderlein et al. \cite{doderlein2022piloting} extended the study of Nguyen et al. \cite{nguyen2022empirical} and run an empirical study on the effect of varying temperature and prompts on the generated code with Copilot and Codex. They used a total of 446 questions to solve from Leetcode and Human Eval data set. 
%
Nathalia et al. \cite{nathalia2023artificial} evaluated the Performance and Efficiency of ChatGPT compared to beginners and experts software engineers. 
%
Yeticstiren et al. \cite{yeticstiren2023evaluating} compared the code quality generated from Copilot, CodeWhisperer, and ChatGPT, showing an advantage for ChatGPT in generating correct solutions. 
%
Guo et al. \cite{guo2023exploring} ran an empirical study on ChatGPT and its capabilities in refining code based on code reviews. 
%
Fu et al. \cite{fu2023chatgpt} also evaluated ChatGPT and its ability to detect, classify, and repair vulnerable code. 
%
Finally, Kabir et al. \cite{kabir2023empirical} evaluated ChatGPT ability to generate code and to maintain it by improving it based on a new feature description to add in the code. 
%
All the above studies focused on either evaluating the ability of LLMs to generate qualitative code, refining it, repairing it if vulnerable, or augmenting it. However, none of them specifically explored the task of code co-evolution.

Moreover, other studies focused on evaluating LLMs in MDE activities. 
\red{Chen et al. \cite{10344012} propose a comparative study between GPT-3.5 and GPT-4 in automatically generating domain models. This work shows that GPT-4 has better modeling results.}
Chaaben et al. \cite{chaaben2023towards} showed how using few-shot learning with GTP3 model can be effective in model completion and in other modeling activities. 
%
Camara et al. \cite{camara2023assessment} further assessed how good ChatGPT is in generated UML models.
%
Finally, Abukhalaf \cite{AbukhalafHK23} run an empirical study on the quality of generated OCL constraints with Codex. %They used 15 UML models and 168 specifications of constraints. 
%
However, these studies also focused on the ability of LLMs to generate MDE artifacts, such as models and constraints, but not on their co-evolution. 
%In addition, most of these studies focus on generation part of LLMs. 
Only Fu et al. \cite{fu2023chatgpt} looked at repairing vulnerable code with ChatGPT. 
%
Jiang et al. \cite{jiang2023selfevolve} proposed self-augmented code generation framework based on LLMs called SelfEvolve. SelfEvolve allows generating code and keep correcting it iteratively with the LLM. % to have better generated code. % but does not treat the metamodel and code coevolution problem.
%
Zhang et al. \cite{zhang2023multilingual} proposed Codeditor, an LLM based tool for code co-evolution between different programming languages. It learns code evolutions as edit sequences and then uses LLMs for multilingual translation. % a  evolution translation from a programming language to another by modeling edit sequences and then learning how to apply them.

To the best of our knowledge, no study investigated the ability of LLMs in the MDE problem of code co-evolution when metamodels evolve. We empirically evaluated how effective is \LLM in solving this co-evolution problem. 

\section{Conclusion}
\label{sec_conclusion}
This paper proposes an automated tracing of the impacted tests due to metamodel evolution. Thus, by tracing the tests before and after code co-evolution, we check its behavioral correctness. 
Our approach takes as input the metamodel changes and then finds the different pattern usages of each metamodel element in the code. 
After that, we recursively search for its usages in the code call graph until reaching the tests. Thus, we end up matching metamodel changes with impacted code methods and their corresponding tests. 
%
We further implemented our approach in an Eclipse plugin that allows to trace the tests, map them with state-of-the-art solution GumTree and execute them. We then report them back as a diagnostic to the developers for an easier in-depth analysis of the effect of metamodel evolutions rather than re-running and analyzing the whole test suite.

\red{The user study experiment we conducted showed that tracing manually the tests impacted by the evolution of the metamodel is a hard and error-prone task. Not only the participants could not trace all tests, but they even wrongly traced non-impacted tests.  
We then evaluated our approach on 18 Eclipse projects from OCL, Modisco, Papyrus, and EMF over several evolved versions of metamodels. Four projects had manually written tests and we generate tests for the other 14 projects. 
The results show that we successfully traced the impacted tests automatically by selecting 1608 out of 34612 tests due to 473 metamodel changes. 

%We evaluated our approach on three implementations of OCL, Modisco and Papyrus in Eclipse on 14 projects from over several evolved versions of metamodels, and with generated tests with EvoSuite. Results show that we automatically traced 1408 out of 33840 tests based on the 466 metamodel changes. %928 and 480 impacted tests 
%
When running the traced tests before and after co-evolution, we observed two cases indicating possibly both behaviorally incorrect and correct code co-evolution. Thus, helping the developers to locate the code co-evolution to investigate in more detail. Furthermore, our approach provided gains that represent, on average, a reduction of 88\% in number of tests and \red{84\%} in execution time. No significant difference was observed between projects with manually written tests and automatically generated ones.}   

As future work, we first plan to improve the performance of our implementation with optimization of the tests' tracing.  %first evaluate on more case studies that could have manually written test. 
%we will 
We also plan to extend our approach to projects that use an equivalent form of metamodels in other technological space than Eclipse, such as JHipster and OpenAPI that both generate code from a model specification similar to a metamodel. Thus, we can have alternative case studies. 

After that, we plan to investigate the techniques of test amplification on the selected tests we traced from the metamodel changes. Indeed, once we select a subset of tests, we could amplify them by generating more similar tests, yet, with different assertions to cover more corner cases. This would amplify the behavioral check of the code co-evolution. 

Finally, we will also explore another type of amplification, which is the interchange of the tests between the original and evolved versions. In other words, we aim to co-evolve the tests of the original and evolved versions, respectively forward and backward to the evolved and original versions, while removing duplicates. 

%%Moreover, the knowledge about the generated code elements from the metamodel elements (e.g., getter/setter for EAttribute, class/interface for EClass, etc.) is so far hard-coded in the implementation. The mappings must be provided for our approach to be able to trace the tests. We could express these mappings in a generic way as a configuration file taken as input in our approach. Thus, we could reuse our approach more easily in other scenarios than EMF Eclipse (e.g., jHipster), given their corresponding mappings as input. This is left as future work to generalize our approach.


%evaluate existing approaches of automatic code co-evolution with out technique to better assess their effect on the code. 

\section*{Acknowledgment.} 
%The research leading to this paper received funding from \emph{Anonym} research project under grant $n^{o}$ \emph{Anonym}.
The research leading to these results has received funding from the \emph{RENNES METROPOLE} under grant \emph{AIS no. 19C0330} and from \emph{ANR} agency under grant \emph{ANR JCJC MC-EVO$^{2}$ 204687}.


\bibliography{bibliography}
\section*{About the authors}
\shortbio{Zohra kaouter Kebaili}{is a PhD student at the IRISA research lab in the DiverSE team, University of Rennes 1, France. Her current research interests are in model-driven engineering, software engineering, and software evolution.\authorcontactf[]{zohra-kaouter.kebaili@irisa.fr}}

\shortbio{Djamel-Eddine Khelladi}{is a CNRS researcher at the IRISA research lab in the DiverSE team, University of Rennes 1, Rennes, France. His current research interests are Software engineering, Model-Driven Engineering, Software Evolution, Co-evolution, Empirical Software Engineering, Incremental Build, Scaling Code Analysis, and Software Processes.
He is a senior member of the ACM.\authorcontact[]{djamel-eddine.khelladi@irisa.fr}}

\shortbio{Mathieu Acher}{is Professor at University of Rennes/IRISA/Inria, France. His research focuses on modelling, reverse engineering, and learning (deep) variability of software-intensive systems. Beyond its applicability, his research is original in combining software engineering and artificial intelligence techniques (symbolic reasoning, machine learning, generative AI). He is the author of more than 150 peer-reviewed publications in international journals and conferences. His work has received Most Influential Paper Award (SLE’19) and Best Paper Awards (SPLC’13, ICPE’19, SPLC’21, ICSR’22, MODELS’23, AST'24). Since 2021, he is a junior research fellow at Institut Universitaire de France (IUF).\authorcontact[ https://mathieuacher.com/]{mathieu.acher@irisa.fr} %for more information.
}

\shortbio{Olivier Barais}{is a full professor of software engineering at the University of Rennes 1 and heads the INRIA DiverSE team in Rennes. With a career dedicated to advancing the field of software engineering, Dr. Barais has made significant contributions to model-driven engineering, software architecture, adaptive systems and open-source software supply chain security. In his role at INRIA, Dr. Barais leads several high-impact research projects and collaborates with industry partners to bring cutting-edge solutions to real-world challenges. Outside of his professional endeavors, he is passionate about mentoring students and fostering collaborative environments that encourage innovation and excellence in software engineering.\authorcontact[https://olivier.barais.fr]{olivier.barais@irisa.fr} %for more information.
}
%\section*{About the authors}
%\shortbio{Alfonso Pierantonio}{is professor at the Università degli Studi dell'Aquila (Italy) and Editor-in-Chief of the Journal of Object Technology. \editorcontact[http://pieranton.io]{alfonso.pierantonio@univaq.it}}
%\shortbio{Mark van den Brand}{is professor at the Technical University of Eindhoven (The Netherlands) and Deputy Editor-in-Chief of the Journal of Object Technology. \editorcontact[]{m.g.j.v.d.brand@tue.nl}}
%\shortbio{Benoit Combemale}{is professor at the University of Toulouse (France) and Deputy Editor-in-Chief of the Journal of Object Technology. \editorcontact[]{benoit.combemale@irisa.fr}}
%\onecolumngrid
\end{document}